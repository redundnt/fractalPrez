\usepackage{mathtools}
\usepackage{accents}
\usepackage{amsmath}
\usepackage{amssymb}
\usepackage{amsthm}
\usepackage{centernot}
\usepackage{color}
\usepackage{enumerate}
\usepackage{geometry}
\usepackage{hyperref}
\usepackage{listings}
\usepackage{tabularx}
\usepackage{mathrsfs}
\usepackage{bbm}
\usepackage{subcaption}
\usepackage[table]{xcolor}
%PAGE FORMATTING
% \addtolength{\voffset}{-.4in}
% \addtolength{\hoffset}{0in}
% \addtolength{\footskip}{5pt}
% \addtolength{\textheight}{100pt}
\addtolength{\textwidth}{-20pt}
%% Shortcuts
\def\R{\mathbb R}
\def\RS{\mathbb S}
\def\Z{\mathbb Z}
\def\Q{\mathbb Q}
\def\C{\mathbb C}

%THEOREMS
\theoremstyle{definition}
\newtheorem{prb}{Problem}
\newtheorem*{prob}{\textbf{Problem :}}
\newtheorem*{prf}{Proof}
\newtheorem*{lem}{Lemma}
\newtheorem*{sln}{Solution}
\newtheorem*{fct}{Fact}
\newtheorem*{dfn}{Definition}
\newtheorem*{dfns}{Definitions}
\theoremstyle{theorem}
\newtheorem{cor}{Corollary}
\newtheorem{thm}{Theorem}
\newcommand{\irr}{\text{irr}}
\newcommand{\ol}[1]{\overline{#1}}
%NEW COMMANDS
\newcommand{\mc}[1]{\mathcal{#1}}
\newcommand{\bb}[1]{\mathbb{#1}}
\newcommand{\bbm}[1]{\mathbbm{#1}}
\newcommand{\ms}[1]{\mathscr{#1}}
\newcommand{\ttt}[1]{\texttt{#1}}
\newcommand{\includecode}[2][Python]{\lstinputlisting[caption=#2, escapechar=, style=custom#1]{#2}}
\newcommand{\embf}[1]{\textbf{\emph{#1}}}
\newcommand{\tbf}[1]{\textbf{#1}}
	% TOPOLOGY COMMANDS
\newcommand{\intr}[1]{\accentset{\circ}{#1}}
\newcommand{\bndr}{\partial}
\newcommand{\clsr}[1]{\overline{#1}}
\newcommand{\tpl}[1][]{\mathscr{T}_{#1}}
	% COMPLEX VARIABLE COMMANDS
\newcommand{\Log}{\text{Log}}
\newcommand{\partials}[2]{\frac{\partial #1}{\partial #2}}
\newcommand{\hessian}[3]{\left(\begin{array}{cc}\frac{\partial^2 #1}{\partial #2^2} & \frac{\partial^2 #1}{\partial #1 \partial #2} \\ \frac{\partial^2 u}{\partial #1\partial #2} & \frac{\partial^2 #1}{\partial #2^2}\end{array}\right)}
\newcommand{\Res}{\text{Res}}
%LISTING

\definecolor{mygreen}{rgb}{0,0.3,0.1}
\definecolor{mygray}{rgb}{0.5,0.5,0.5}
\definecolor{mymauve}{rgb}{0.58,0,0.82}
\definecolor{purple}{rgb}{0.78,0,0.82}
\definecolor{TEAL}{rgb}{0.2, 0.32, 0.5}
\definecolor{MAROON}{rgb}{0.55,0.051,0.12}
\definecolor{BEIGE}{rgb}{0.8,0.90,0.5}
\definecolor{AQUA}{rgb}{0.0, 0.8, 0.8}
\definecolor{GRAY90}{gray}{0.95}
\definecolor{GRAY80}{gray}{0.85}

\lstset{ %
  backgroundcolor=\color{white},   % choose the background color; you must add \usepackage{color} or \usepackage{xcolor}
  basicstyle=\footnotesize,        % the size of the fonts that are used for the code
  breakatwhitespace=false,         % sets if automatic breaks should only happen at whitespace
  breaklines=true,                 % sets automatic line breaking
  captionpos=b,                    % sets the caption-position to bottom
  commentstyle=\itshape\color{mygreen},
  deletekeywords={...},            % if you want to delete keywords from the given language
  escapeinside={\%*}{*)},          % if you want to add LaTeX within your code
  extendedchars=true,              % lets you use non-ASCII characters; for 8-bits encodings only, does not work with UTF-8
  frame=double,                    % adds a frame around the code
  keepspaces=true,                 % keeps spaces in text, useful for keeping indentation of code (possibly needs columns=flexible)
  keywordstyle=\bfseries\color{blue},       % keyword style
  language=Python,                 % the language of the code
  otherkeywords={*,...},            % if you want to add more keywords to the set
  numbers=left,                    % where to put the line-numbers; possible values are (none, left, right)
  numbersep=8pt,                   % how far the line-numbers are from the code
  numberstyle=\tiny\color{black}, % the style that is used for the line-numbers
  rulecolor=\color{black},         % if not set, the frame-color may be changed on line-breaks within not-black text (e.g. comments (green here))
  showspaces=false,                % show spaces everywhere adding particular underscores; it overrides 'showstringspaces'
  showstringspaces=false,          % underline spaces within strings only
  showtabs=false,                  % show tabs within strings adding particular underscores
  stepnumber=1,                    % the step between two line-numbers. If it's 1, each line will be numbered
  stringstyle=\color{mymauve},     % string literal style
  tabsize=2,                       % sets default tabsize to 4 spaces
  title=none                   % show the filename of files included with \lstinputlisting; also try caption instead of title
}

% LISTING STYLES
% customPython
\lstdefinestyle{customPython}{
  backgroundcolor=\color{BEIGE},
  belowcaptionskip=1\baselineskip,
  breaklines=true,
  frame=lines,
  numbers=left,
  xleftmargin=\parindent,
  language=Python,
  showstringspaces=false,
  basicstyle=\footnotesize\ttfamily,
  keywordstyle=\ttfamily\color{blue},
  commentstyle=\ttfamily\color{mygreen},
  identifierstyle=\color{TEAL},
  stringstyle=\color{MAROON},
  caption=
}
\lstdefinestyle{customPythonSnippit}{
  backgroundcolor=\color{BEIGE},
  belowcaptionskip=1\baselineskip,
  breaklines=true,
  frame=none,
  %xleftmargin=\parindent,
  language=Python,
  showstringspaces=false,
  basicstyle=\footnotesize\ttfamily,
  keywordstyle=\ttfamily\color{blue},
  commentstyle=\ttfamily\color{mygreen},
  identifierstyle=\color{TEAL},
  stringstyle=\color{MAROON},
  numbers=none,
  caption=
}
% customBash
\lstdefinestyle{customBash}{
  backgroundcolor=\color{BEIGE},
  belowcaptionskip=1\baselineskip,
  breaklines=true,
  frame=lines,
  xleftmargin=\parindent,
  language=Bash,
  showstringspaces=false,
  basicstyle=\footnotesize\ttfamily,
  keywordstyle=\ttfamily\color{blue},
  commentstyle=\ttfamily\color{mygreen},
  identifierstyle=\color{blue},
  stringstyle=\color{blue},
  caption=
}
% href setup
\hypersetup{pdfborder={0 0 0}, urlcolor=blue, linkcolor=blue, colorlinks=true}
